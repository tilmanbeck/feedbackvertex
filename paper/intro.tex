\chapter{Einleitung}
\label{c:intro} % Always give a unique label
In dieser Arbeit wird die Cut \& Count-Technik aus \cite{cygan_solving_2011} behandelt und eine konkrete Implementierung für das Steinerbaum-Problem dargestellt. 
In Sektion \ref{sec:intro_not} werden wir die Notation einführen, welche für das Verständnis der folgenden Kapitel wichtig ist. 
Die Cut \& Count-Technik benutzt eine angepasste Form einer Nice Tree Decomposition aus \cite{kloks1994}, welche wir in Kapitel \ref{c:ntd} definieren und veranschaulichen.
In Kapitel \ref{c:cc_general} wird die Funktionsweise der Cut \& Count-Technik allgemein erklärt. 
Anschließend wird in Kapitel \ref{c:cc_steiner} die Technik auf das Steinerbaum-Problem angewendet und erläutert. 
Kapitel \ref{c:impl} umfasst unsere Implementierung zum Steinerbaum-Problem, eine kurze Evaluation zu verschiedenen Eingabegrößen, die Diskussion unserer Ergebnisse und einen Ausblick. 
Im letzten Kapitel \ref{c:summary} wird der Inhalt dieser Arbeit zusammengefasst.

\section{Notation}
\label{sec:intro_not}
Für den Rest der Arbeit bedienen wir uns der Notation aus der Arbeit \cite{cygan_solving_2011}. 
Die Bezeichnung $G=(V,E)$ beschreibt einen ungerichteten Graphen. Entsprechend beschreiben $V(G)$ und $E(G)$ die Menge der Knoten bzw. Kanten des Graphen $G$. 
Die Bezeichung $G[X]$ einer Knotenmenge $X \subseteq V(G)$ steht für den Subgraphen, der von $X$ erzeugt wird. Für eine Menge an Kanten $X \subseteq E$ beschreibt $V(X)$ die Menge der Endknoten der Kanten aus $X$ und $G[X]$ den Subgraphen $(V,X)$. 
Die Knotenmenge für eine Menge von Kanten $X$ im Graphen $G[X]$ ist diesselbe wie im Graphen $G$.

Mit einem \glqq Schnitt\grqq ~einer Menge $X \subseteq V$ ist das Paar $(X_1,X_2)$ mit den Eigenschaften $X_1 \cap X_2 = \emptyset,X_1 \cup X_2 = X$ gemeint. 
$X_1$ und $X_2$ werden als linke und rechte \glqq Seiten\grqq des Schnittes bezeichnet. 

Eine Zusammenhangskomponente beschreibt eine Teilmenge eines Graphen, die zusammenhängend ist.

Die Zahl $cc(G)$ eines Graphen $G$ beschreibt die Anzahl der Zusammenhangskomponenten (\glqq \textbf{c}onnected \textbf{c}omponents\grqq).

Für zwei Bags $x,y$ eines Baums mit Wurzelknoten gilt, dass $y$ ein Nachkomme von $x$ ist, falls es möglich ist ausgehend von $y$ einen Weg zu $x$ zu finden, der im Baum nur in Richtung des Wurzelknotens verläuft. 
Insbesondere ist $x$ sein eigener Nachkomme.

Für zwei Integer $a,b$ sagt die Gleichung $a \equiv b$ aus, dass $a$ genau dann gerade ist, wenn auch $b$ gerade ist.
Zudem wird Iverson's Klammernotation verwendet. 
Falls $p$ ein Prädikat ist, dann sei $[p]$ 1 (0) falls $p$ wahr (unwahr) ist.
Falls $\omega:U\rightarrow {1,\dots,N}$, so bezeichnet $\omega(S)=\sum_{e\in S} \omega(e)$ für $S \subseteq U$.

Für eine Funktion $s$ mit $s[v \rightarrow \alpha]$ schreiben wir die Funktion $s \ {(v,s(v))}\cup{(v,\alpha)}$. Diese Definition funktioniert unabhängig davon, ob $s(v)$ bereits definiert wurde oder nicht.
