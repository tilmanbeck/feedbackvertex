%%%%%%%%%%%%%%%%%%%%% chapter.tex %%%%%%%%%%%%%%%%%%%%%%%%%%%%%%%%%
%
% sample chapter
%
% Use this file as a template for your own input.
%
%%%%%%%%%%%%%%%%%%%%%%%% Springer-Verlag %%%%%%%%%%%%%%%%%%%%%%%%%%
%\motto{Use the template \emph{chapter.tex} to style the various elements of your chapter content.}
\chapter{Cut \& Count für das Steinerbaum-Problem}
\label{c:cc_steiner}

\section{Steinerbaum}
\label{sec:steiner}
\begin{definition}
Steinerbaum-Problem\\
\textbf{Eingabe}: Sei $G = (V, E)$ ein ungerichteter Graph, $T \subseteq V$ eine Menge von Terminalknoten und $k$ eine positive Ganzzahl. \\
\textbf{Problemstellung}: Gibt es eine Menge $X \subseteq V$ der Kardinalität $k$, so dass $T \subseteq X$ und $G[X]$ zusammenhängend ist?
\end{definition}

In einem gegebenen Graphen ist eine Menge $T \subseteq V$ als Terminale markiert. Gesucht ist ein Subgraph innerhalb des Ursprungsgraphen, der alle Terminale und genau $k$ Knoten enthält. Zudem muss der Subgraph zusammenhängend sein, es muss also jeder Knoten von jedem anderen Knoten des Subgraphen über einen Pfad innerhalb des Subgraphen erreichbar sein.

\section{Cut}
\label{sec:st_cut}
Zu Beginn des Cut-Teils wird eine zufällige Gewichtsfunktion $\omega:V\rightarrow \{1,\dots,N\}$ definiert. 
Diese wird für die Isolation der Lösungsmenge verwendet. 
$\omega$ weist jedem Knoten zufällig ein Gewicht zu. 
Wird $\omega$ auf eine Menge von Knoten angewendet, ist das Ergebnis die Summe der einzelnen Knotengewichte.\\
Anschließend definieren wir die Menge $\mathcal{R}$, welche die Zusammenhangs-Bedingung abschwächt. 
Somit ist $\mathcal{R}_W$ die Menge aller Teilmengen von $X$ aus $V$ mit $T \subseteq X$, $\omega(X)=W$ und $|X|=k$. Die Menge $\mathcal{R}_W$ beschreibt alle Lösungskandidaten.\\
Die Menge $\mathcal{S}_W=\{X \in \mathcal{R}_W | G[X]$ ist zusammenhängend$\}$ umfasst die Lösungsmenge für eine Menge $\mathcal{R}_W$. 
$\cup_W \mathcal{S}_W$ bildet so die gesamte Lösungsmenge. 
Gibt es ein Gewicht $W$ für das die Menge nicht leer ist, so gibt der Algorithmus eine positive Antwort.\\
Von der Menge der Terminalknoten wird ein zufälliges Terminal als $v_1$-Terminal festgelegt. 
Dieses dient dazu, dass bei der Bildung von konsistenten Schnitten kein Schnitt doppelt gezählt wird.\\
Ein Schnitt $(V_1,V_2)$ eines ungerichteten Graphen $G=(V,E)$ ist konsistent, falls $u \in V_1$ und $v \in V_2$ impliziert, dass $uv \notin E$. 
Ein Subgraph, der einen konsistenten Schnitt aus $G$ bildet, ist ein Paar $(X,(X_1,X_2))$, so dass $(X_1,X_2)$ ein konsistenter Schnitt von $G[X]$ ist.
$\mathcal{C}_W$ sei die Menge aller Subgraphen, die einen konsistenten Schnitt $(X,(X_1,X_2))$ bilden, wobei $X\in \mathcal{R}_W$ und $v_1 \in X_1$.

Die Anzahl der Subgraphen, welche einen konsistenten Schnitte bilden, sind in Lemma 3.3 in \cite{cygan_solving_2011} definiert:\\

Let $G=(V,E)$ be a graph and let $X$ be a subset of vertices such that $v_1 \in X \subseteq V$. The number of consistently cut subgraphs $(X,(X_1,X_2))$such that $v_1 \in X_1$ is equal to $2^{cc(G[X])-1}$.\\

Ausgehend von der Definition $\mathcal{C}_W$ ist für jeden Subgraphen, der einen konsistenten Schnitt $(X,(X_1,X_2))$ bildet, und für jede Zusammenhangskomponente $C$ aus $G[X]$ bekannt, dass  $C$ entweder in $X_1$ oder in $X_2$ enthalten sein muss. Für die Zusammenhangskomponente, die $v_1$ enthält, ist die Seite des Schnitts fest. Für alle anderen Zusammenhangskomponenten kann die Seiten der Mengen $X_1,X_2$ frei gewählt werden. Daher erhalten wir $2^{cc(G[X])-1}$ verschiedene konsistente Schnitte.

\begin{figure}
  \centering
    \includegraphics[width=0.5\textwidth]{./imgs/terminal_v1.png}
  	\caption{Konsistente Schnitte}
	\label{fig:st_cut}
\end{figure}

In der Abbildung \ref{fig:st_cut} wird an einem Beispiel die Bildung von konsistenten Schnitten dargestellt. Die getrichelte und gepunktete Linie veranschaulicht die Grenzen der konsistenten Schnitte. Die Knoten mit doppeltem Rand stellen Terminale da. Das Terminal $A$ wird als $v_1$ Terminal festgelegt. Zur Vereinfachung wurde die Gewichte der Knoten ignoriert.
Aus diesem Beispiel gehen mit $k=3$ folgende Mengen hervor:
\begin{itemize}
\item $\mathcal{R} = \{\{A,B,C\}, \{A,B,D\}, \{A,B,E\}\}$
\item $\mathcal{S} = \{\{A,B,C\}\}$
\end{itemize}

Aus der $\mathcal{R}$ Menge ergeben sich folgende $\mathcal{C}$ Mengen:
\begin{itemize}
\item Für $\mathcal{R}_1$:
\begin{itemize}
\item $\mathcal{C}_{1.1} = (X=(A,B,C),  (X_1=\{A,B,C\}, X_2=\{\emptyset\}))$
\end{itemize}
\item Für $\mathcal{R}_2$:
\begin{itemize}
\item $\mathcal{C}_{2.1} = (X=(A,B,D),  (X_1=\{A,B\}, X_2=\{D\}))$
\item $\mathcal{C}_{2.2} = (X=(A,B,D),  (X_1=\{A,B,D\}, X_2=\{\emptyset\}))$ 
\end{itemize}
\item Für $\mathcal{R}_3$:
\begin{itemize}
\item $\mathcal{C}_{3.1} = (X=(A,B,E),  (X_1=\{A,B\}, X_2=\{E\}))$
\item $\mathcal{C}_{2.1} = (X=(A,B,D),  (X_1=\{A,B,E\}, X_2=\{\emptyset\}))$
\end{itemize}
\end{itemize}

Ohne die Restriktion $v_1 \in X_1$ gäbe es die doppelte Menge an konsistenten Schnitten. Bei jedem Schnitt könnte die Mengen $X_1$ und $X_2$ vertauscht werden. Durch die Restriktion von $v_1$ wird dies vermieden.


\section{Count}
\label{sec:st_count}
Aus Lemma 3.3 ist bekannt: $|\mathcal{C}|=\sum_{X \in \mathcal{R}} 2^{cc(G[X])-1}$. \\
Durch die Randomisierung mithilfe der Knotengewichte wird die Wahrscheinlichkeit, dass die Menge der Lösungskandidaten $\mathcal{R}$ mehrere Lösungen enthält, reduziert. 
Für eine Lösung $\mathcal{S} \in \mathcal{R}$ gilt, dass $G[X]$ zusammenhängend ist. 
Durch die Festlegung des $v_1$-Terminals wird zudem die Möglichkeit des Schnitts für eine Lösung auf eins reduziert (Schnitt mit der leeren Menge). 
Daher gilt $|\mathcal{S}| = |\{X \in \mathcal{R}| cc(G[X]=1\}|$.
Also können wir schreiben: $|\mathcal{C}_W| \equiv |\mathcal{S}_W|$.

Diese Äquivalenz erlaubt es ein dynamisches Programm zu beschreiben, dass nicht die Menge der Lösungen $\mathcal{S}$, sondern die Anzahl der Subgraphen $|\mathcal{C}_W|$, die einen konsistenten Schnitt bilden, zu berechnen. Dieses Programm ist in folgender Sektion näher erläutert.
%Wir legen $W$ fest und ignorieren die Indices: $|\mathcal{C}| \equiv |\{X \in \mathcal{R} |cc(G[X]) = 1\}| = |\mathcal{S}|$. 
%Das bedeutet, dass die Anzahl der konsistenten Schnitte eines Graphen modulo zwei gleich der Anzahl der Lösungen ist. 
%Im Lemma 3.4 trifft das Paper dazu eine Aussage: Let $G,$ $\omega$, $\mathcal{C}_W$ and $\mathcal{S}_W$ be as defined above. 
%Then for every $W$, $|\mathcal{S}_W| \equiv |\mathcal{C}_W|$.

\section{Dynamisches Programm}
\label{sec:dynP}

Für das dynamische Programm werden für jeden Bag $x \in \mathbb{T}$, die Zahlen $0 \leq i \leq k,0 \leq w \leq kN$ und die Färbung $s \in \{0,1_1,1_2 \}^{B_x}$ folgende Mengen definiert:
\begin{itemize}
\item $\mathcal{R}_x(i,w)=\{X \subseteq V_x | (T \cap V_x) \subseteq X$ $\wedge$ $|X| = i$ $\wedge$ $\omega (X) = w \}$ \\
beschreibt die Menge der Lösungskandidaten
\item $\mathcal{C}_x (i,w) =\{ (X,(X_1,X_2)) | X \in \mathcal{R}_x(i,w)$ $\wedge$ $(X,(X_1,X_2))$ ist ein Subgraph, der einen konsistenten Schnitt von $G_x$ bildet $\wedge$ $(v_1 \in V_x \Rightarrow v_1 \in X_1 \} $
\item $\mathcal{A}_x(i,w,s)=| \{ (X,(X_1,X_2)) \in \mathcal{C}_x(i,w) | (s(v) = 1_j \Rightarrow v \in X_j)$ $\wedge$ $(s(v)=0 \Rightarrow v \notin X \} |$ \\
beschreibt die Anzahl der Mengen $\mathcal{C}_x(i,w)$, wobei die in $X$ enthaltenen Knoten mit $s \in \{1_1,1_2\}$ gefärbt sind und die restlichen Knoten mit $s=0$ 
\end{itemize}
Die Färbung gibt an, ob und zu welcher Seite eines konsistenten Schnitts ein Knoten gehört.
Die Zuordnung der Färbung $s \in \{0,1_1,1_2 \}^{B_x}$  der Knoten aus Bag $B_x$ bzgl. der Menge $C_x$ ist folgendermaßen definiert:
\begin{itemize}
\item $s[v] = 0 \Rightarrow v \notin X$
\item $s[v] = 1_1 \Rightarrow v \in X_1$ 
\item $s[v] = 1_2 \Rightarrow v \in X_2$ 
\end{itemize}
$A_x(i,w,s)$ zählt alle Möglichkeiten die Knoten eines Bags $x$ gemäß der Definition zu färben. Es gibt für einen Bag $3^{B_x}$ Möglichkeiten die Knoten zu färben.

Im dynamischen Programm werden die folgenden Berechnungsregeln für jede $A_x(i,w,s)$ Matrix eines Bags $x$ angewandt. Zur Vereinfachung der Notation beschreibt im folgenden $v$ den neu eingeführten Knoten. $y$ und $z$ stehen für das linke bzw. das rechte Kind:
\begin{itemize}
\item \textbf{Leaf bag}:
\begin{itemize}
\item $A_x=(0,0,\emptyset) = 1$\\Leaf bags enhalten keine Knoten, daher werden sie mit einem Initialwert gefüllt.
\end{itemize}
\item \textbf{Introduce Vertex v}:
\begin{itemize}
\item $A_x=(i,w,s[v\rightarrow 0]) = [v \notin T]A_y(i,w,s)$\\ Ist der eingeführte Knoten kein Terminal, so wird der Wert aus dem Bag des Kindes übernommen.
\item $A_x=(i,w,s[v\rightarrow 1_1]) = A_y(i-1,w-w(v),s)$\\ Reduziere beim Zugriff auf den Bag des Kindes i um 1 und ziehen das Gewicht des eingeführten Knoten ab. Übernehme den Wert.
\item $A_x=(i,w,s[v\rightarrow 1_2]) =[v \neq v_1] A_y(i-1,w-w(v),s)$\\ Ist der eingeführte Knoten nicht das speziell gewählte Terminal, so verfahre wie bei $1_1$.
\end{itemize}
\item \textbf{Introduce Edge uv}
\begin{itemize}
\item $A_x(i,w,s) = [s(u) = 0 \vee s(v) = 0 \vee s(u) = s(v)]A_y(i,w,s)$\\ Ist einer der Knoten $0$ gefärbt oder sind beide gleich gefärbt, so wird der Wert aus dem Bag des Kindes übernommen.
\end{itemize}
\item \textbf{Forget Vertex v}
\begin{itemize}
\item $A_x(i,w,s) = \sum\limits_{\alpha \in {0,1_1,1_2}} A_x(i,w,s[v \rightarrow \alpha]) $\\ Es wird die Summe  über alle Färbungen des vergessenen Knoten im Bag des Kindes gebildet.
\end{itemize}
\item \textbf{Join Bag}
\begin{itemize}
\item $A_x(i,w,s) = \sum\limits_{i_1+i_2=i+|s^{-1}({1_1,1_2})|}$   $\sum\limits_{w_1+w_2=w+w(s^{-1}({1_1,1_2}))} A_y(i_1,w_1,s)A_z(i_2,w_2,s) $\\Die innere Summe iteriert über die Gewichte der Knoten innerhalb der Bags der beiden Kinder. Ist deren Summe gleich der Summe von $w$ und der Summe der Gewichte von Knoten mit der Färbung $1_1$ und $1_2$, so werden sie akkumuliert.
\\Die äußere Summe iteriert über die Laufvariable $i$ der Bags der Kinder. Ist die Summe gleich der Summe von $i$ und der Anzahl der Knoten die $1_1$ und $1_2$ gefärbt sind, so werden sie akkumuliert.
\end{itemize}
\end{itemize}


Ob der Algorithmus eine Lösung gefunden hat, kann aus der $k \times kN$-Datenmatrix des Wurzel-Knotens $A_r(i,w,\emptyset)$ ausgelesen werden. 
Dieser Zugriff ist konstant in $\mathcal{O}(1)$.\\
Falls eine Lösung existiert und diese gefunden wurde, dann existiert ein $0 \leq W \leq kN$ für das $A_r(k,W,\emptyset) \equiv 2 = 1$.

\section{Monte-Carlo Algorithmus und Laufzeit}
\label{sec:mc_alg}
Im Theorem 3.6 aus \cite{cygan_solving_2011} wird zusammenfassend erwähnt:
\begin{theorem}
There exists a Monte-Carlo algorithm that given a tree decomposition of width $t$ solves STEINER TREE in $3^t|V|^{\mathcal{O}(1)}$ time. The algorithm cannot give false positives and may give false negatives with probability at most 1/2.
\end{theorem}

Die Laufzeit setzt sich wie folgt zusammen, wobei $t$ für die Baumweite der NTD steht.
\begin{itemize}
\item $3^t$:\\ Für jeden Bag muss über alle Farbkombinationen iteriert werden. Die obere Grenze hierbei ist die Anzahl der Knoten im größten Bag. Dieser besitzt $3^t$ verschiedene Färbungen.
\item $|V|^{\mathcal{O}(1)}$:\\ obere Grenze der Eingabe-Parameter k und N.
\item Die Wahrscheinlichkeit von 1/2 für falsch-negativ entsteht durch die Gewichtsfunktion $\omega:V\rightarrow \{1,\dots,N\}$ und durch das Isolations-Lemma, sofern $N=2|V|$ gesetzt wird. 
\end{itemize}