\chapter{Cut \& Count-Technik}
\label{c:cc_general}

\section{Einführendes}
\label{sec:cc_intro}
Die Cut \& Count-Technik aus \cite{cygan_solving_2011} wird benutzt um Zusammenhangs-Probleme von Graphen zu lösen.
Dabei handelt es sich um Graphen-Probleme, bei denen zusammenhängende Submengen der Knoten gefunden werden müssen, die problemspezifische  Eigenschaften erfüllen müssen. 
Die Technik setzt sich aus den beiden Teilen \glqq Cut\grqq ~, \glqq Count\grqq ~zusammen (siehe Sektion \ref{sec:cc_cc}) und kann auf mehrere Probleme angewendet werden, u.A. Längster Weg, Steinerbaum oder Feedback Vertex Set.

Das Ergebnis ist ein Monte-Carlo-Algorithmus, welcher eine Aussage über die Existenz einer (oder mehrerer) Lösungen gibt. Dies ist in Sektion \ref{sec:cc_monte} erläutert.
Mithilfe von Randomisierung beschränkt die Technik die Anzahl der Lösungsmengen mit einer Wahrscheinlichkeit, die durch das Isolation-Lemma begrenzt werden kann. Dies wird in Sektion \ref{sec:cc_iso} näher erklärt.


\section{Cut \& Count}
\label{sec:cc_cc}
Sei $\mathcal{S}$ die Menge der Lösungen für das Graphenproblem. 
Dann entscheidet Cut \& Count ob diese Menge leer ist. 
Dazu kann Cut \& Count folgendermaßen aufgeteilt werden:

\subsection{Cut}
\label{ssec:cc_cut}
Die Zusammenhangs-Bedingung aus dem Graphenproblem wird abgeschwächt, indem eine Menge von Lösungskandidaten $\mathcal{R} \supseteq \mathcal{S}$ betrachtet wird. 
Zusätzlich wird die Menge $\mathcal{C}$ von Paaren $(X,C)$ betrachtet, wobei $X \in \mathcal{R}$ und $C$ einen konsistenten Schnitt beschreibt. 

\subsection{Count}
\label{ssec:cc_count}
Berechne $|\mathcal{C}|$ modulo 2 mithilfe der Prozedur $\mathtt{countC}$ (siehe Sektion \ref{ssec:impl_countc}). 
Dadurch werden alle unzusammenhängenden Lösungskandidaten $X \in \mathcal{R} \setminus \mathcal{S}$ ignoriert, da die Anzahl derer mit einer geraden Anzahl von konsistenten Schnitten konsistent sind. Es bleiben die zusammenhängenden Lösungskandidaten $x \in \mathcal{S}$ übrig. 

Der Algorithmus \glqq zählt\grqq ~nicht die Lösungen, sondern die Anzahl der Lösungskandidaten, die einen konsistenten Schnitt bilden.

\section{Monte-Carlo-Algorithmus}
\label{sec:cc_monte}
Ein Monte-Carlo-Algorithmus ist ein randomisierter Algorithmus, welcher mit einer nach oben beschränkten Wahrscheinlichkeit ein falsches Ergebnis liefern darf. 
Für die Cut \& Count-Technik bedeutet dies, dass der hervorgegangene Algorithmus niemals ein falsch-positives Ergebnis\footnote{Der Algorithmus gibt eine positive Antwort auf das Problem, obwohl keine Lösung existiert} ausgeben kann, aber zu einer gewissen Wahrscheinlichkeit ein falsch-negatives Ergebnis\footnote{Der Algorithmus gibt eine negative Antwort auf das Problem, obwohl eine Lösung existiert.}. 
Die Wahrscheinlichkeit eines falsch-negativ wird durch das Isolations-Lemma begrenzt. 
Die Ausgabe eines falsch-negativ unterliegt einer bestimmten Wahrscheinlichkeit. Daher ist sinnvoll den Algorithmus in diesem Fall mehrmals hintereinander ausführen. Es begünstigt die Wahrscheinlichkeit eine Lösungsmenge zu finden, falls diese existiert.

Der Monte-Carlo-Algorithmus erreicht folgende Laufzeiten für einige bekannte Zusammenhangs-Probleme bei Graphen, wobei $t$ die Braumbreite der Tree Decomposition beschreibt:
\begin{itemize}
\item Steinerbaum in $3^t |V|^{O(1)}$
\item Feedback Vertex Set in $3^t|V|^{O(1)}$
\item ...
\end{itemize}

\section{Isolation Lemma}
\label{sec:cc_iso}
Da im Count-Teil die Menge $|\mathcal{C}|$ modulo 2 berechnet wird, muss sichergestellt werden, dass die Menge der Lösungen begrenzt wird. 
Für eine gerade Anzahl von Lösungen würde der Algorithmus sonst ein falsch-negatives Ergebnis liefern, da jede gerade Anzahl durch die Modulo-Operation nicht erkannt wird. 
Daher muss die Lösungsmenge reduziert werden. Dafür wird auf das Isolations-Lemma zurückgegriffen. Dieses ist wie folgt definiert:

\begin{definition}
Isolation Lemma\\
 A function $\omega : U \rightarrow \mathbb{Z}$ isolates a set family $\mathcal{F} \subseteq 2^U$ if there is a unique $S' \in \mathcal{F}$ with $\omega (S')=min_{S \in \mathcal{S}} \omega(S)$\\
\end{definition}

Die Funktion $\omega : U \rightarrow \mathbb{Z}$ weißt jedem Element $U$ einen Zahlenwert unabhängig und zufällig zu.
Bei der Anwendung auf das Steiner-Tree-Problem steht das $U$ stellvertretend für einen Knoten.
Die Funktion $\omega$ heißt dabei Gewichtsfunktion und die zugewiesene Zahl ist das Gewicht des Knoten.
Die Set-Familie $\mathcal{F}$ ist die Menge aller Knoten.
Gibt es nun genau ein Subset $S'$ aus $\mathcal{F}$ mit mit der Eigenschaft $\omega (S')=min_{S \in \mathcal{S}} \omega(S)$, so sagt man das $\omega$ $\mathcal{F}$ isoliert.

In der Arbeit \cite{cygan_solving_2011} wird das Isolations-Lemma in Lemma 2.5 verwendet, um folgende Aussage über die Wahrscheinlichkeit der Isolation auszusagen:\\
\\Let $\mathcal{F} \subseteq 2^U$ be a set family over a universe $U$ with $|\mathcal{F}| > 0$. For each $u \in U$ ,
choose a weight $\omega(u) \in {1, 2, . . . , N }$ uniformly and independently at random. Then
\begin{center}
$prob[\omega$ isolates $\mathcal{F}]\geq 1 - \frac{|U|}{N}$
\end{center}

Durch die Wahl eines großen $N$ kann die Wahrscheinlichkeit eines falsch-negativ reduziert werden. Dies geht mit einer Erhöhung der Laufzeit des Algorithmus einher.
