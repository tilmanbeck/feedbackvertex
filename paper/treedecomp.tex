%%%%%%%%%%%%%%%%%%%%% chapter.tex %%%%%%%%%%%%%%%%%%%%%%%%%%%%%%%%%
%
% sample chapter
%
% Use this file as a template for your own input.
%
%%%%%%%%%%%%%%%%%%%%%%%% Springer-Verlag %%%%%%%%%%%%%%%%%%%%%%%%%%
%\motto{Use the template \emph{chapter.tex} to style the various elements of your chapter content.}
\chapter{(Nice) Tree Decomposition}
\label{c:ntd} % Always give a unique label
% use \chaptermark{}

\section{Tree Decomposition}
\label{sec:ntd_td}
\begin{definition}
(Tree decomposition, \cite{robertson1984}). Eine Tree Decomposition \textit{eines (ungerichteten oder gerichteten) Graphen $G$ ist ein Baum $\mathbf{T}$ in dem jedem Knoten $x \in \mathbf{T}$ eine Menge von Knoten $B_x \subseteq V$ (genannt \glqq Bag\grqq) zugeordnet ist, so dass }
\end{definition}

\section{Nice Tree Decomposition}
\label{sec:ntd_ntd}

\section{Further Requirements}
\label{sec:ntd_req}
