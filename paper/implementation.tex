\chapter{Implementierung}
\label{c:impl}

\section{Einführung}
\label{sec:impl_intro}
In unserer Implementierung liegt der Fokus auf der Implementierung des dynamischen Programms der Cut \& Count-Technik. 
Wir nehmen an, dass für den Ursprungsgraphen $G$ bereits eine standardmäßige Nice Tree Decomposition gemäß Sektion \ref{sec:ntd_ntd} vorliegt. 
In Sektion \ref{sec:impl_ntd} erklären wir, wie die standardmäßige Nice Tree Decomposition zu einer Nice Tree Decomposition (siehe Sektion \ref{sec:ntd_req}) erweitert wird. 
Anschließend erläutern wir in Sektion \ref{sec:impl_dynP} die Implementierung des dynamischen Programms und gehen auf implementierungsspezifische Details ein. 
In Sektion \ref{sec:impl_eval} zeigen und diskutieren wir unsere Ergebnisse der Laufzeit mit verschiedenen Eingabegröße.
Sektion \ref{sec:impl_outlook} gibt einen Ausblick für mögliche Optimierungen und Erweiterungen unserer Implementierung.

\section{Nice Tree Decomposition}
\label{sec:impl_ntd}
Es wurde ein Algorithmus entwickelt, der als Eingabe eine standardmäßige Nice Tree Decomposition $\mathbb{T}$ eines Graphen $G$ erhält und eine Nice Tree Decomposition (siehe Kapitel \ref{sec:ntd_ntd}) ausgibt. 
Da der Fokus dieser Arbeit auf der Implementierung des dynamischen Programms des Cut \& Count-Algorithmus liegt, wurde der Algorithmus nicht hinsichtlich der in \cite{kloks1994} beschriebenen polynomiellen Laufzeit optimiert. 

Der Algorithmus iteriert mehrmals in symmetrischer Reihenfolge (ausgehend von der Wurzel linksseitig absteigend) über $\mathbb{T}$ und fügt dabei die fehlenden Knoten ein. 
Hierbei sollte erwähnt werden, dass in unserer Implementierung mit Ausnahme des \glqq Join\grqq -Bags neue Knoten stets linksseitig an den Elternknoten angehängt werden. 
Dies ist für die rekursive Iteration in Sektion \ref{sec:impl_dynP} wichtig.

Zu Beginn werden am bisherigen Wurzelknoten so lange \glqq Forget\grqq -Knoten angehängt bis noch ein Knoten des Ursprungsgraphen im Bag verbleibt. 
Anschließend wird ein letzter Knoten mit leerem Bag als neuer Wurzelknoten hinzugefügt. 
Ähnlich wird hinsichtlich der Blattknoten verfahren. 
Entsprechend der Differenz eines leeren Bags und der Bags der bisherigen Blattknoten werden am Ende jedes Pfades neue \glqq Introduce-Vertex\grqq -Knoten und ein Knoten mit leerem Bag als neuer Blattknoten angehängt. 
Für bestehende \glqq Join\grqq -Knoten werden die Bags der beiden Kindknoten verglichen. 
Sofern sie nicht denselben Bag wie der \glqq Join\grqq -Knoten haben, werden neue Knoten (\glqq Forget\grqq , \glqq Introduce Vertex\grqq ) eingefügt, bis die Bags identisch mit dem Elternknoten sind. 
In der nächsten Iteration werden die Differenzen der Bags von Kind- und Elternknoten verglichen. 
Falls diese größer eins ist, werden entsprechend viele neue Knoten (\glqq Forget\grqq , \glqq Introduce Vertex\grqq ) eingeführt.
Zuletzt wird für jede Kante $e$ des Ursprungsgraphen $G$ über den Graphen iteriert. 
Beim ersten gemeinsamen Auftreten der Knoten der Kante $e$ innerhalb eines Bags, wird oberhalb des Knoten dieses Bags ein neuer \glqq Introduce Edge\grqq -Knoten eingeführt und mit den Knoten der Kante $e$ gekennzeichnet.

Nach diesen Modifikationen liegt der Graph in der Form einer Nice Tree Decomposition wie in Sektion \ref{sec:ntd_ntd} beschrieben vor und kann zur Berechnung des dynamischen Programms verwendet werden.

\section{Dynamisches Programm}
\label{sec:impl_dynP}
Die Berechnungsvorschrift des dynamischen Programms ist in Sektion \ref{c:cc_steiner} erläutert. 
Für die Implementierung wird vom Wurzelknoten ausgehend in symmetrischer Reihenfolge über die Nice Tree Decomposition $\mathbb{T}$ iteriert und für jeden Bag eine $k \times kN \times 3^{|B_x|}$ - Matrix berechnet, wobei die Größe der Lösung $k$ und $N$ als Eingabeparameter übergeben werden. 
Die erste Dimension $k$ beschreibt die Größe (Anzahl Knoten) der Lösungsmenge. 
Die zweite Dimension $kN$ steht für die Summe der Gewichte der Lösungsmenge. 
Obwohl die Gewichte zufällig einheitlich verteilt werden, kann im schlechtesten Fall jedem Knoten das Gewicht $N$ zugewiesen werden. 
Daher kann die Gesamtsumme der Gewichte $W$ gleich $kN$ sein.
$3^{|B_x|}$ beschreibt die Anzahl der möglichen Färbungen innerhalb eines Bags. 
Während $k,N$ festgelegt sind, kann die Länge der Farb-Dimension $3^{|B_x|}$ von Bag zu Bag variieren. 
\subsection{Implementierung der CountC-Prozedur}
\label{ssec:countc}
Da das dynamische Programm des Cut \& Count-Algorithmus die Berechnungsvorschrift für einen Bag rekursiv über den Bag des jeweiligen Kind-Knotens definiert, steigt der Algorithmus zu Beginn in $\mathbb{T}$ rekursiv ab bis er an einem Blattknoten angekommen ist. 
Für diesen gibt es keine Farb-Dimension (der Bag ist leer) und die $k \times kN$-Matrix wird initialisiert. 
Anschließend wird beim rekursiven Aufstieg für jeden Bag die entsprechende Berechnungsvorschrift angewendet und eine neue Datenmatrix berechnet. 

Im Falle eines \glqq Introduce Vertex\grqq -Bag werden für jede Färbung des Bags des Kindknotens drei neue Färbungen hinzugefügt.
Für einen \glqq Forget\grqq -Bag werden jeweils drei Färbungen des Bags des Kindknotens zu einer Färbung zusammengeführt.
Für alle anderen Bag-Typen bleibt die Länge der Farb-Dimension gleich, es werden jedoch nur Werte übernommen, welche die in Sektion \ref{sec:dynP} definierten Bedingungen erfüllen. 
Der \glqq Join\grqq -Bag nimmt eine Sonderstellung ein, da er als einziger zwei Kindknoten besitzt und im Algorithmus auf die Daten beider Bags zugreift. 
Die rekursive Berechnung in symmetrischer Reihenfolge gewährleistet, dass beide Kindknoten eines \glqq Join\grqq -Bags vorher berechnet werden. 
Der letzte Berechnungsschritt für den Wurzelknoten entspricht der Berechnung eines \glqq Forget\grqq -Knoten und führt die letzten drei Färbungen zusammen, so dass der Wurzelknoten (mit leerem Bag) eine $k \times kN$-Datenmatrix enthält. 
Diese kann für die Abfrage der Lösungen $A_r(k,W,\emptyset)$ genutzt werden.

Der Programmablauf ist im Algorithmus \ref{code:pseudo} dargestellt.

\begin{svgraybox}
\begin{lstlisting}[language=python,label={code:pseudo}, basicstyle=\small, caption=Pseudocode für das dynamische Programm]
v1 = terminals[0]
def countC(node, indices, child_data, k, N, terminals, weights):
    if node.linker_Knoten is not None:
        # steige linksseitig ab
        child_data = countC(node.linker_Knoten, indices, child_data, k, N, terminals, weights)
    if node.rechter_Knoten is not None:
	# steige rechtsseitig ab
	child_data_right = countC(node.rechter_Knoten, indices, child_data, k, N, terminals, weights)
    if node.Bag_Typ == Leaf:
    	# erzeuge und initialisiere k x kN Datenmatrix
    	dm = new data[k, k*N]
    	dm[0,0] =  1
    	return dm
    elif node.Bag_Typ == Root:
	# v wurde aus Bag entfernt
	data = new data[3^node.bag.length, k, k*N]
	for i in range(0,k):
		for w in range(0,kN):
			data[s,i,w] = child_data[s=0,i,w] + child_data[s=1,i,w] + child_data[s=2,i,w]
	return data			
    elif node.Bag_Typ == Introduce_Edge:
	# im 'Introduce Edge'-Bag (u,v sind die Knoten der Kante e)
	# s ist die Faerbung
		data = data = new data[3^node.bag.length, k, k*N]
		for s in len(data):
			if( s(u) == 0 | s(v) == 0 | s(u) == s(v) ):
				for i in range(0,k):
					for w in range(0,kN):
						data[s,i,w] = child_data[s,i,w]
        return data
    elif node.Bag_Typ == Introduce_Vertex:
        # erzeuge neue Datenmatrix (groesser als Kind, da pro Knoten im Bag drei neue Faerbungen)
        # v ist der 'introduced vertex' 
        data = new data[3^(child_data.length + 1), k, k * N]
        
        for s in child_colorings:
        	# berechne Indices fuer neue Faerbungen
        	indices = calculate_new_indices
            for i in range(0, k):
                for w in range(0, k * N):
                    if not terminals.contains(v):
                        data[indices[0], i, w] = child_data[s, i, w]
                    data[indices[1], i, w] = child_data[s, i - 1, w - weights(v)]
                    if v != v1:
                        data[indices[2], i, w] = data[s, i - 1, w - weights.get(introduced_vertex)]
        return data
    elif node.Bag_Typ == Forget:
    	# v ist der 'forget vertex'
        data = new data[3^(child_data.length - 1), k, k * N]
		
        for s in colorings:
            for i in range(0, k):
                for w in range(0, k * N):
                    # berechne die Indices der drei Faerbungen, die aufsummiert werden
                    indices = calculate_indices()
                    data[s, i, w] = child_data[indices[0], i, w] +
                    		    child_data[indices[1], i, w] +
                                    child_data[indices[2], i, w]
        return data
    elif node.Bag_Typ == Join:
	data = new data[node.bag.length, k, k * N]
        for s in colorings:
            for i in range(0, k):
                for w in range(0, k * N):
                    colored_nodes = all_nodes_by_coloring(1,2)
                    bound_i = i + colored_nodes.length
                    bound_w = w + get_sum_of_weights(colored_nodes)
                    for i1 in range(0, bound_i):
                        for w1 in range(0, bound_w):
                            i2 = acc_bound_1 - i1
                            w2 = acc_bound_2 - w1
                            if w1 >= (k * N) or w2 >= (k * N) or i1 >= k or i2 >= k:
                                value += 0
                            else:
                                value += child_data[s, i1, w1] *   
                                         child_data_right[s, i2, w2])
                    data[s, i, w] = value
        return data
    return data

\end{lstlisting}
\end{svgraybox}

\subsection{Berechnung der Färbungen}
\label{ssec:colors}
Für die Kodierung der Färbungsdimension muss eine geeignete Zuordnung von Färbungen zu Indices der Datenmatrix eingeführt werden. Dafür benutzen wir eine ternäre Kodierung mit folgender Zuordnung:
\begin{center}
$0 \rightarrow s=0$\\
$1 \rightarrow s=1_1$\\
$2 \rightarrow s=1_2$\\
\end{center}
Die ternäre Darstellung beschreibt die Färbung einer Menge von Knoten, wobei jede Ziffer einen gefärbten Knoten markiert. 
Die entsprechende Dezimaldarstellung steht für den Index der Färbungsdimension. 
Der Zugriff auf diesen Index liefert die $k \times kN$-Datenmatrix für die entsprechende Färbung.
Für die Reduzierung bzw. Erweiterung der Färbungsdimension werden Funktionen benötigt, welche die richtige Zuordnung zwischen Färbung und Kodierung gewährleisten.
Dazu wurde die Funktion \texttt{calculate\_indices} implementiert. 
Diese bekommt als Parameter die Färbung als geordnete Liste (z.~B. \texttt{[0,2,1]} für 7 als Index), wobei jede Zahl dieser Liste für einen entsprechend der o.g. Zuordnung gefärbten Knoten steht. 
Zusätzlich wird eine Liste von Indices übergeben, welche die \glqq freien\grqq ~Stellen der Färbungsliste markieren. 
Die \glqq freien\grqq ~Stellen sind die Knoten, für die alle Färbungsmöglichkeiten mit den anderen Stellen der Liste kombiniert werden. 
Die Funktion berechnet die entsprechenden Möglichkeiten mittels Horner-Schema und gibt sie als null-basierte Dezimalzahlen zurück. 
Diese dienen im dynamischen Programm als Zugriffsindices für die Färbungsdimension der Datenmatrizen. 
Als Beispiel wird der Funktion \texttt{([0,2], [0])} als Parameter übergeben. 
Dies bedeutet, dass alle Färbungsmöglichkeiten der ersten Stelle (null-basiert) mit der zweiten Stelle (mit der Färbung $2 \rightarrow 1_2$) kombiniert werden. 
Als Ergebnis gibt die Funktion in diesem Fall die Liste \texttt{[2,5,8]} als Ergebnis aus, wobei \texttt{[2,5,8]} für die Färbungen \texttt{[0,2],[1,2],[2,2]} stehen.

Um auf Färbungen einzelner Knoten zurückgreifen zu können (z.~B. für die Berechnung des \glqq Introduce Edge\grqq -Bags) wird eine Funktion \texttt{get\_index\_as\_list} benötigt. Dieser wird eine Färbung als Index (in Dezimaldarstellung) und die Länge des Bags als Parameter übergeben wird. Diese berechnet die Liste der ternär codierten Färbungen mit der Länge des Bags. Der Code dafür ist in Algorithmus \ref{code:as_list} dargestellt. 

\begin{svgraybox}
\begin{lstlisting}[language=python,label={code:as_list}, basicstyle=\small, caption=Funktion get\_index\_as\_list zur Berechnung der ternären Färbung aus dem Index]
def get_index_as_list(x, nr_of_vertices):
    if nr_of_vertices == 0:
        return [0]
    number = x
    res = []
    for i in range(nr_of_vertices-1, -1, -1):
        value = m.floor(number/(3**i))
        number -= (value * 3**i)
        res.append(value)

    return res
\end{lstlisting}
\end{svgraybox}

\section{Evaluierung}
\label{sec:impl_eval}
Die Laufzeit des dynamischen Programms wurde anhand von Beispielen getestet. Zu sehen sind die Eingabeparameter k und N sowie die Anzahl der Knoten des Graphen sowie die in der Nice-Tree-Decomposition erhaltene \textit{treewidth}.
\begin{table}
\centering
\begin{tabular}{l | l} 
\textbf{Eingabe} & \textbf{ $\varnothing$ T in $s$}\\
\hline
(k=2, N= 6, |V|=3, tw(G)=2) & $ \sim 0.002$ \\
(k=3, N=14, |V|=7, tw(G)=3) & $ \sim 0.83$ \\
(k=4, N=32, |V|=16, tw(G)=3) & $ \sim 14.23$ \\
\end{tabular}
\end{table}

Es ist ersichtlich, dass der Algorithmus schon bei kleinen Grapherweiterungen merkbare Laufzeitverlängerungen aufweist. Es ist anzunehmen, dass der Algorithmus in der Praxis nur eingeschränkt nutzbar ist. Praxisrelevante Probleme werden sich wahrscheinlich in Graphen repräsentieren, die weit aus komplexer sind, als die von uns getesteten Beispiele. Wir vermuten die Laufzeit wäre dort nichtmehr tragbar oder zumindest nicht effizient genug. Desweiteren ist der Speicherbedarf der Cut \& Count-Technik nicht außer acht zu lassen. Jeder Bag benötigt eine $3^t\times k\times kN$-Matrix. Bei größeren Graphen und hohem gewählten $N$, zur Verringerung des Falsch-Negativ Ergebnisses, kann diese schnell sehr groß werden.

\section{Ausblick}
\label{sec:impl_outlook}

