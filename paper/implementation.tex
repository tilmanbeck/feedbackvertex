%%%%%%%%%%%%%%%%%%%%% chapter.tex %%%%%%%%%%%%%%%%%%%%%%%%%%%%%%%%%
%
% sample chapter
%
% Use this file as a template for your own input.
%
%%%%%%%%%%%%%%%%%%%%%%%% Springer-Verlag %%%%%%%%%%%%%%%%%%%%%%%%%%
%\motto{Use the template \emph{chapter.tex} to style the various elements of your chapter content.}
\chapter{Implementierung}
\label{c:impl} % Always give a unique label
% use \chaptermark{}
% to alter or adjust the chapter heading in the running head

\section{Nice Tree Decomposition}
\label{sec:impl_ntd}
Es wurde ein Algorithmus entwickelt, der als Eingabe eine standardmäßige Nice Tree Decomposition $\mathbb{T}$ eines Graphen $G$ erhält und eine Nice Tree Decomposition (siehe Kapitel \ref{sec:impl_ntd}) ausgibt. Da der Fokus dieser Arbeit auf der Implementierung des dynamischen Programms des Cut \& Count-Algorithmus liegt, wurde der Algorithmus nicht hinsichtlich der in \cite{kloks1994} beschriebenen polynomiellen Laufzeit optimiert. 

Der Algorithmus iteriert mehrmals in symmetrischer Reihenfolge ( über $\mathbb{T}$ und fügt dabei die fehlenden Knoten ein. Zu Beginn werden am bisherigen Wurzelknoten so lange \glqq Forget\grqq -Knoten angehängt bis noch ein Knoten des Ursprungsgraphen im Bag verbleibt.Anschließend wird ein letzter Knoten mit leerem Bag als neuer Wurzelknoten hinzugefügt. Ähnlich wird hinsichtlich der Blattknoten verfahren. Entsprechend der Differenz eines leeren Bags und der Bags der bisherigen Blattknoten werden neue \glqq Introduce-Vertex\grqq -Knoten eingefügt und am Ende jedes Pfades ein Knoten mit leerem Bag als neuer Blattknoten angehängt. Für bestehende \glqq Join\grqq -Knoten werden die Bags der beiden Kindknoten verglichen. Sofern sie nicht denselben Bag wie der \glqq Join\grqq -Knoten haben, werden neue Knoten (\glqq Forget\grqq , \glqq Introduce Vertex\grqq ) eingefügt, bis die Bags identisch mit dem Elternknoten sind. 
Anschließend wird über den Graphen iteriert
Zuletzt wird für jede Kante $e$ des Ursprungsgraphen $G$ über den Graphen iteriert. Beim ersten gemeinsamen Auftreten der Knoten der Kante $e$ innerhalb eines Bags, wird oberhalb des aktuellen Knoten im Graphen ein neuer \glqq Introduce Edge\grqq -Knoten eingeführt und mit den Knoten der Kante $e$ gekennzeichnet.

\section{Dynamisches Programm}
\label{sec:impl_dynP}

\section{Evaluierung}
\label{sec:eval}

\section{Ausblick}
\label{sec:outlook}
