\section{Zusammenfassung}
\label{c:summary}
Das Ergebnis der Anwendung der Cut \& Count-Technik auf ein Graphproblem ist ein Monte-Carlo-Algorithmus, der mit einer nach oben beschränkten Wahrscheinlichkeit Aussage über die Existenz eines Ergebnisses liefert. Mittels Randomisierung und dem Isolations-Lemma \ref{sec:cc_iso} wird die Wahrscheinlichkeit einer falsch-negativen Antwort nach oben begrenzt. 

In dieser Arbeit wurde diese Technik für das Steinerbaum-Problem implementiert. Da die Cut \& Count-Technik eine angepasste Nice Tree Decomposition benötigt, wurde eine Implementierung entwickelt, welche eine standardmäßige Nice Tree Decomposition in die von der Cut \& Count-Technik benötigte Form überführt.

Die Evaluation verschiedener Eingabegrößen warf die Frage auf inwiefern die Count \& Count-Technik praxisrelevante Probleme hinsichtlich der Laufzeit und des Speicherbedarfs effizient lösen kann. 
Für die Technik ist eine effiziente Vorverarbeitung der Nice Tree Decomposition wichtig, da die Baumweite und die Anzahl der Bags Auswirkungen auf die Laufzeit und den Speicherbedarf haben.
Hierbei bedarf es weiterer Forschung in Vergleich mit anderen Ansätzen, wie z.~B. einer Brute-Force-Suche.